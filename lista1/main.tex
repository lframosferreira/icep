\documentclass{article}

\usepackage[colorlinks=true, allcolors=blue]{hyperref}

\title{Lista 1}
\author{Luís Felipe Ramos Ferreira}
\date{\href{mailto:lframos.lf@gmail.com}{\texttt{lframos.lf@gmail.com}}
}

\begin{document}

\maketitle

\begin{enumerate}

	\item Capítulo lido
	      %imagem de trabalho feito
	\item
	      \begin{itemize}
		      \item (1.5.3)
		            \begin{enumerate}
			            \item papel
			            \item papel
			            \item papel
		            \end{enumerate}
		      \item (1.5.6) Seja \(G\) um grafo qualquer com \(n\) vértices. Suponha, por contradição, que não existam dois vértices em \(G\) com o mesmo grau. Logo, como existem \(n\) vértices no grafo, os \(n\) possíveis graus que um vértice pode ter são \(\{0, 1, \dots, n-1\}\), logo podemos dizer que estes são os graus dos vértices de \(G\). No entanto, isso é absurdo, pois existiram um vértice de grau 0 e um vértice de grau \(n-1\) em um grafo com \(n\) vértices, o que não faz sentido. Logo, a premisa inicial estava errada, e podemos afirmar que todo grafo com \(n\) vértices, \(n \geq 2\), possui dois vértices com o mesmo grau.
		      \item (1.5.11) indução?
	      \end{itemize}
	\item

\end{enumerate}

\end{document}
