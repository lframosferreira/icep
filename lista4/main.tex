\documentclass{article}

\usepackage{amsmath}
\usepackage{amssymb}
\usepackage{float}
\usepackage{graphicx}
\usepackage{tikz}
\usepackage{textpos}

\usepackage[colorlinks=true, allcolors=blue]{hyperref}

\title{Lista 4}
\author{Luís Felipe Ramos Ferreira}
\date{\href{mailto:lframos.lf@gmail.com}{\texttt{lframos.lf@gmail.com}}
}

\begin{document}

\maketitle

\begin{itemize}
	\item (6.7.1)
	\item (6.7.2)

	      Vamos primeiramente relembrar o que é uma família crescente \(\mathcal{A}\) de grafos. Uma família de grafos é crescente se para
	      todo \(G \in \mathcal{A}\), se \(G \subseteq G'\), então \(G' \in \mathcal{A}\).

	      Vamos construir dois modelos \(G(n, p)\), um com probabilidade \(p_1\) e outro com probabilidade \(p_2\), de modo que \(0 \leq p_1 < p_2 \leq 1\).
	      Vamos construir os grafos de maneira natural: escolhemos de maneira uniforme um número real \(r\) aleatório no intervalo \([0, 1]\), para cada par de vértices
	      em um grafo com \(n\) vértices. Se \(r \leq p_1\),  adicionamos à aresta referente ao par de vértices em \(G(n,p_1)\). Se \(r \leq p_2\), adicionamos a aresta
	      em \(G(n, p_2)\).

	      Pela maneira como o grafo foi construído, note que se uma aresta existe em \(G(n, p_1)\), ela também existe em \(G(n, p_2)\), uma vez que \(p_1 < p_2\). Logo, temos certeza
	      que \(G(n, p_1) \subseteq G(n, p_2)\). Pela hipótese inicial, sabemos que \(\mathcal{A}\) é uma família crescente em grafos. Se \(G(n, p_1)\) pertence a \(\mathcal{A}\),
	      então com certeza \(G(n, p_2) \in \mathcal{A}\). Isso nos mostra que \(\mathbb{P}(G(n, p) \in \mathcal{A})\) é uma função que cresce com \(p\), isto é:

	      \[\mathbb{P}(G(n, p_1) \in \mathcal{A}) \leq \mathbb{P}(G(n, p_2) \in \mathcal{A})\]

	      para todo \(p_1 \leq p_2\).

	\item (6.7.5)

	      Queremos mostra que, se \(p << \frac{1}{n}\), então \(G(n, p)\) não contêm ciclos com alta probabilidade. Para fazer isso, vamos mostrar que o valor esperado
	      da quantidade de ciclos em \(G(n,p)\) tende a 0 quando \(p << \frac{1}{n}\). Vamos denotar por \(X\) a variável aleatória que representa a quantidade de ciclos no \(G(n, p)\)
	      e por \(X_k\) a variável aleatória que representa a quantidade de ciclos de tamanho \(k\) em \(G(n, p)\). Temos evidentemente que \(\mathbb{E} = \sum_{k=3}^n \mathbb{X}[X_k]\).

	      Para calcular \(\mathbb{E}[X_k]\), vamos imaginar o seguinte. Dos \(n\) vértices do grafo, temos que escolher \(k\) para formar o ciclo. Portanto, teremos um fator
	      da forma \(\binom{n}{k}\). Para cada uma das arestas desse ciclo existir, precisaremos da probabilidade \(p^k\), pois a existência de cada aresta depende de \(p\).
	      Temos que levar em consideração também as permutações desse vértices, por isso mais um fator de \((k-1)!\) deve ser adicionado (são todas as permutações de 1 até \(n\), mas o primeiro vértice não importa pois um ciclo não
	      tem um "primeiro vértice" como em um caminho, por exemplo, portanto usamos \(k-1\) e não \(k\)). Temos também que dividir por 2 uma vez que a orientação do ciclo não altera a unicidade de um ciclo.
	      \(\{1, 2, 3\}\) e \(\{3, 2, 1\}\) são permutações de vértices que forma o mesmo ciclo, por exemplo.
	      Portanto, \(\mathbb{E}[X_k] = \binom{n}{k}\frac{(k-1)!}{2} p^k\).

	      Sabemos que \(\binom{n}{k} = \frac{n!}{k!(n-k)!} \leq \frac{n^k}{k!}\). Partindo disso, podemos afirmar que \(\mathbb{E}[X_k] = \binom{n}{k}\frac{(k-1)}{2}!p^k \leq \frac{n^k}{k!}\frac{(k-1)!}{2}p^k\).

	      Após alguns algebrismos podemos notar que \(\mathbb{E}[X_k] \leq \frac{(np)^k}{2k}\). Como \(p << \frac{1}{n}\), temos que \(\frac{p}{1/n} = np\) tende a 0. Desse modo,
	      \(\mathbb{E}[X_k] \leq \frac{0^k}{k}\) quando \(p << \frac{1}{n}\). Como, para cada \(k\), temos que \(\mathbb{E}[X_k]\) tende a 0, o somatório \(\sum_{k=3}^n \mathbb{E}[X_k]\) também tende a 0,
	      e concluímos que se \(p << \frac{1}{n}\), com alta probabilidade, \(\mathbb{E}[X] = 0\), isto é, a quantidade de ciclos no grafo \(G(n, p)\) é zero, o que faz do grafo
	      acíclico, como queríamos demonstrar.




\end{itemize}

\end{document}
