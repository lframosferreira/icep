\documentclass{article}

\usepackage{amsmath}
\usepackage{amssymb}
\usepackage{float}
\usepackage{graphicx}
\usepackage{tikz}
\usepackage{textpos}

\usepackage[colorlinks=true, allcolors=blue]{hyperref}

\title{Lista 4}
\author{Luís Felipe Ramos Ferreira}
\date{\href{mailto:lframos.lf@gmail.com}{\texttt{lframos.lf@gmail.com}}
}

\begin{document}

\maketitle

\begin{itemize}
	\item (6.7.1)
	\item (6.7.2)

	      Vamos primeiramente relembrar o que é uma família crescente \(\mathcal{A}\) de grafos. Uma família de grafos é crescente se para
	      todo \(G \in \mathcal{A}\), se \(G \subseteq G'\), então \(G' \in \mathcal{A}\).

	      Vamos construir dois modelos \(G(n, p)\), um com probabilidade \(p_1\) e outro com probabilidade \(p_2\), de modo que \(0 \leq p_1 < p_2 \leq 1\).
	      Vamos construir os grafos de maneira natural: escolhemos de maneira uniforme um número real \(r\) aleatório no intervalo \([0, 1]\), para cada par de vértices
	      em um grafo com \(n\) vértices. Se \(r \leq p_1\),  adicionamos à aresta referente ao par de vértices em \(G(n,p_1)\). Se \(r \leq p_2\), adicionamos a aresta
	      em \(G(n, p_2)\).

	      Pela maneira como o grafo foi construído, note que se uma aresta existe em \(G(n, p_1)\), ela também existe em \(G(n, p_2)\), uma vez que \(p_1 < p_2\). Logo, temos certeza
	      que \(G(n, p_1) \subseteq G(n, p_2)\). Pela hipótese inicial, sabemos que \(\mathcal{A}\) é uma família crescente em grafos. Se \(G(n, p_1)\) pertence a \(\mathcal{A}\),
	      então com certeza \(G(n, p_2) \in \mathcal{A}\). Isso nos mostra que \(\mathbb{P}(G(n, p) \in \mathcal{A})\) é uma função que cresce com \(p\), isto é:

	      \[\mathbb{P}(G(n, p_1) \in \mathcal{A}) \leq \mathbb{P}(G(n, p_2) \in \mathcal{A})\]

	      para todo \(p_1 \leq p_2\).

	\item (6.7.5)




\end{itemize}

\end{document}
