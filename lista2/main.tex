\documentclass{article}

\usepackage{amsmath}
\usepackage{float}
\usepackage{graphicx}

\usepackage[colorlinks=true, allcolors=blue]{hyperref}

\title{Lista 2}
\author{Luís Felipe Ramos Ferreira}
\date{\href{mailto:lframos.lf@gmail.com}{\texttt{lframos.lf@gmail.com}}
}

\begin{document}

\maketitle

\begin{itemize}
	\item (4.5.1)
		\begin{itemize}
			\item \(R(3)\)
				Inicialmente, note que a seguinte 2-coloração do \(K_5\) não possui uma clique de tamanho 3 monocromática, portanto
				\(R(3) > 5\). 

				No entanto, sabemos pelo fato 4.0.1 do livro que o toda 2-coloração do \(K_6\) possui um triângulo monocromático, logo \(R(3) = 6\).
				A prova funciona da seguinte forma: seja \(v\) um vértice de \(K_6\). Pelo princípio da casa dos pombos, das 5 arestas incidentes a \(v\),
				ao menos 3 possuem a mesma cor. Vamos dizer que é a cor 1. Sejam \(x, y, z\) vizinhos de \(v\) com a aresta com cor 1. Se qualquer uma das arestas
				\(xy, xz, yz\) for da cor 1, temos um triângulo de cor 1. Caso contrário, o triângulo formado pelos vértices \(x, y, z\) é monocromático na outra cor,
				chamemos ela de 2. Logo, toda 2-coloração do \(K_6\) possui um triângulo monocromático.
			\item \(R(3, 4)\)
				Inicialmente, vamos notar que \(R(3, 4)\) é maior que 8, e isso pode ser notado pela 2-coloração do \(K_8\) abaixo em que
				não existe uma clique de tamanho 3 vermelha e nem uma clique de tamanho 4 azul.
				COLOCAR K8 COLORIDO AQ
				Vamos provar agora que \(R(3, 4) \leq 10\), em particular, mostrar que para um grafo completo de 10 vértices sempre teremos um triângulo
				vermelho ou uma clique de tamanho 4 azul. Depois, com uma pequena variação, mostraremos que \(R(3,4) \leq 9\), o que conclui a prova.

				Seja \(A\) um vértice qualquer de um \(K_10\) 2-colorido com vermelho e azul. \(A\) possui nove vizinhos e das arestas que o conectam a seus vizinhos,
				sabemos que ao menos 6 são azuis ou ao menos 4 são vermelhas (isso porque no total precisamos ter 9 arestas, uma para cada vizinho). Suponhamos o caso em que
				\(A\) possui 4 arestas vermelhas o conectando a seus vizinhos. Se existir uma aresta vermelha entre esses vizinhos, então existe um triângulo vermelho no grafo. Caso
				contrário, todas as arestas entre os 4 vértices são azuis, logo existe uma clique de tamanho 4 de cor azul. Seja agora o caso em que \(A\) possui
				6 arestas de cor azul o conectando a seus vizinhos. Sabemos que \(R(3, 3) = 6\), logo, entre esses vizinhos, há um triângulo vermelhor ou azul. Se for vermelho,
				já perdemos, se for azul, note que ele forma uma clique de tamanho 4 azul junto com \(A\). Logo, \(10\) é um limite superior para \(R(3, 4)\).
				
				Consideremos agora o caso do \(K_9\). Note que os argumentos usados anteriormente servem da mesma maneira, exceto pelo caso em que, para todo vértice \(A\),
				exista exatamente 5 arestas azuis e 3 vermelhas saindo dele. Nesse caso, para cada vértice teremos três arestas vermelhas, e como são 9 vértices, temos \(3 * 9 = 27\). Como cada arestas
				é contada duas vezes, precisamos dividir por dois, obtendo assim um número \(\frac{27}{2}\)(não inteiro) de arestas, o que é um absurdo. Logo, \(R(3, 4) \leq 9\)

			\item \(R(4, 4)\)
				 Sabemos pelo lema 4.1.3 do livro que, para todo \(s, t \geq 2\), temos:
				 \[R(s, t) \leq R(s - 1, t) + R(s, t - 1)\]

				 Logo, temos que \(R(4, 4) \leq R(3, 4) + R(4, 3) = 2*R(3, 4) = 2*9 = 18\)
				 No entanto, vamos mostrar que existe uma 2-coloração de \(K_{17}\) tal que não existe uma clique de tamanho 4 nem vermelha nem azul, mostrando assim
				 que \(R(4, 4) = 18\).
		\end{itemize}

	\item (4.5.2)
	\item (4.5.3)
	\item (4.5.4)
	\item (4.5.5)
	\item (4.5.6)
	\item (4.5.7)
	\item (4.5.8) OPCIONAL
	\item (4.5.9)
	\item (4.5.10)
\end{itemize}

\end{document}
