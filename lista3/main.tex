\documentclass{article}

\usepackage{amsmath}
\usepackage{amssymb}
\usepackage{float}
\usepackage{graphicx}
\usepackage{tikz}
\usepackage{textpos}

\usepackage[colorlinks=true, allcolors=blue]{hyperref}

\title{Lista 3}
\author{Luís Felipe Ramos Ferreira}
\date{\href{mailto:lframos.lf@gmail.com}{\texttt{lframos.lf@gmail.com}}
}

\begin{document}

\maketitle

\begin{itemize}
	\item (5.8.1)

	      Vamos escolher os valores de \(a_i\) da seguinte maneira:
	      \begin{itemize}
		      \item \(a_i = -1\) com probabilidade \(\frac{1}{2}\)
		      \item \(a_i = 1\) com probabilidade \(\frac{1}{2}\)
	      \end{itemize}

	      Isso claramente implica que, para todo \(i, j\), o produto \(a_i a_j\) segue a seguinte distribuição:

	      \begin{itemize}
		      \item \(a_i a_j = -1\), com probabilidade \(\frac{1}{2}\)
		      \item \(a_i a_j = 1\), com probabilidade \(\frac{1}{2}\)
	      \end{itemize}

	      A esperança de \(a_i a_j\), denotada por \(\mathbb{E}[a_i a_j]\), é igual à \(-1 * \frac{1}{2} + 1 * \frac{1}{2} = 0\). Usando a linearidade
	      da esperança, temos que:

	      \[\mathbb{E}[a_i a_j] = \]

	\item (5.8.2)

	      Queremos provar o teorema de Turán, que diz que se um grafo \(G\) com \(n\) vértices é livre de \(K_r\), então:

	      \[|E(G)| \leq (1 - \frac{1}{r-1})\frac{r^2}{2}\]

	      Sabemos, a partir do teorema 5.3.11 do livro (que devemos usar na prova), que todo grafo \(G\) possui em sua estrutura uma clique de tamanho \(\sum_{v \in V(G)} \frac{1}{n - d(v)}\),
	      onde \(d(v)\) é o grau de \(v\) em \(G\). Isso pois todo conjunto independente em \(G\) é uma clique em \(\overline{G}\).

	      Vamos inicialmente então assumir que o grafo \(G\) é livre de \(K_r\), a clique com \(r\) vértices. Disso podemos assumir que:

	      \[\sum_{v \in V(G)} \frac{1}{n - d(v)} \leq r -1\]

	      Como \(\frac{1}{n - d(v)}\) é uma função convexa em relação a \(d(v)\), podemos aplicar a transformação:

	      \[\frac{n}{n - \sum_{v \in V(G)} \frac{d(v)}{n}} \leq r - 1\]
	      \[\frac{n}{n - \frac{2 |E(G)|}{n}} \leq r - 1\]
	      \[\frac{n}{r - 1} \leq n - \frac{2 |E(G)|}{n}\]
	      \[\frac{2 |E(G)|}{n} \leq n - \frac{n}{r - 1}\]
	      \[\frac{2 |E(G)|}{n} \leq n (1 - \frac{1}{r - 1})\]
	      \[|E(G)| \leq \frac{n^2}{2} (1 - \frac{1}{r - 1})\]

	      Como queríamos demonstrar.

	\item (5.8.3)

	      Para resolver essa questão também usaremos o teorema 5.3.11 do livro. Seja \(G\) um grafo com \(n\) vértices e \(\pi = [v_1, \dots, v_{n}]\) uma permutação
	      aleatória e uniforme dos vértices de \(G\). Denotamos por \(\mathcal{N}(v)\) a vizinhança aberta de \(v\) em \(G\). Vamos construir um conjunto \(S\) da seguinte maneira:

	      \begin{itemize}
		      \item \(S \gets \emptyset\)
		      \item \(\forall v \in \pi\)
		            \begin{itemize}
			            \item se \(\mathcal{N}(v) \subsetneq S\):
			                  \begin{itemize}
				                  \item \(S \gets S \cup \mathcal{N}(v)\)
			                  \end{itemize}
		            \end{itemize}
	      \end{itemize}

	      Como \(G\) é um grafo livre de triângulo, como diz o enunciado, podemos afirmar que o conjunto \(S\) é independente. Isso porque

	      Seja agora \(X\) a variavel aleatória que segue a seguinte regra:

	      \begin{itemize}
		      \item \(X(v) = 1\) se \(v \in S\)
		      \item \(X(v) = 0\) c.c.
	      \end{itemize}

	      Temos que o tamanho do conjunto \(S\) é tal que \(|S| = \sum_{v \in V(G)} X_v\). O valor esperado do tamanho desse conjunto é portanto:

	      \[\mathbb{E}[|S|] = \mathbb{E}[\sum_{v \in V(G)} X_v]\]

	      Pela linearidade da esperança, temos que:

	      \[\mathbb{E}[\sum_{v \in V(G)} X_v] = \sum_{v \in V(G)} \mathbb{E}[X_v] = \sum_{v \in V(G)} \mathbb{P}(v \in S)\]

	\item (5.8.4)
	\item (5.8.5)
	\item (5.8.6)
	\item (5.8.7)
\end{itemize}

\end{document}
